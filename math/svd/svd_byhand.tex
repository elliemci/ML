\documentclass[14pt]{article}

\usepackage{amsmath}
\usepackage[colorlinks=true, allcolors=blue]{hyperref}
\usepackage[margin=1in]{geometry} % Set the margin to 1 inch

\title{Single Value Decomposition of a Mattrix}
\author{Ellie}

\begin{document}
\maketitle

\section{Compute the $\mathbf{A^T}$  and $\mathbf{A^TA}$ for the given matrix}
 
  $$
    A = \left[\begin{array}{ll}
    4 & \phantom{-}0  \\
    3 & -5 
    \end{array}\right]
  $$

  $$
  \mathbf{A^T} = \left[\begin{array}{ll}
  4 & 3  \\
  0 & -5 
  \end{array}\right]
  $$

\section{Find the eigenvalues of $\mathbf{A^TA}$, sort them in descending order}

    \begin{flalign*}
    &\mathbf{A^TA} = \left[\begin{array}{ll}
    4 & 3  \\
    0 & -5 
    \end{array}\right] \left[\begin{array}{ll}
    4 & 0  \\
    3 & -5 
    \end{array}\right] = \left[\begin{array}{ll}
    \phantom{-}25 & -15  \\
    -15 & \phantom{-}25 
    \end{array}\right] 
    = (25-\lambda)^2 - 15^2  = (25-\lambda - 15)(25 - \lambda+ 15)
    \end{flalign*}

    \begin{flalign*}
    = (10-\lambda)(40-\lambda) = 0 \\
    \lambda_1 = 40, \lambda_2 = 10
    \end{flalign*}

\section{Construct $\Sigma$ as a diagonal matrix of the singular values = square roots of the eigenvalues of 
$\mathbf{A^TA}$ and find $\Sigma^{-1}$}
$$
\Sigma = \left[\begin{array}{ll}
\sqrt{40}  & 0  \\
0 & \sqrt{10}  
\end{array}\right]
$$

Since $\Sigma$ is diagonal, its inverse is simply the reciprocal of the diagonal elements:

\begin{flalign*}
&\Sigma^{-1} = \left[\begin{array}{ll}
\frac{1}{\sqrt{40}}  & 0  \\
0 & \frac{1}{\sqrt{10}} 
\end{array}\right] = \left[\begin{array}{ll}
0.1581 & 0  \\
0 & 0.3162
\end{array}\right]
\end{flalign*}

\section{Constrruct $\mathbf{V}$ with columns the unit eigenvectors of $\mathbf{A^TA}$}

\begin{flalign*}
\boldsymbol{\lambda = 40}: 
\end{flalign*}

$$
(\mathbf{A} - 40\mathbf{I})\vec{X} = \left[\begin{array}{ll}
(25-40) & -15  \\
-15 & (25-40) 
\end{array}\right]\begin{bmatrix}
x_1 \\
x_2
\end{bmatrix} = \mathbf{0}
$$

Solving the system of equations:
    \begin{equation}
        \begin{cases}
          -15x_1 - 15x_2 &= 0 \\
          -15x_1 - 15x_2 &= 0
        \end{cases}
    \end{equation}

which results in $x_2 = -x_1$. Thus, for any real number $x_1$, a vector of the form
    $\vec{X} = \left[\begin{array}{r}
            \phantom{-}x_1  \\
            -x_1
           \end{array}\right]$

is an eigenvector of $\mathbf{A^TA}$ corresponding to $\lambda = 40$.
To make it an unit length divide by its magnitude  \\
$$\left\lVert X \right\rVert = 
\sqrt{x_1^2 + x_2^2} = \sqrt{2x_1^2} = \sqrt{2}x_1
$$

Hence, $\hat{X} = \left[\begin{array}{l}
                       \phantom{-}\frac{1}{\sqrt{2}}  \\
                      -\frac{1}{\sqrt{2}} 
                       \end{array}\right]$
is the unit eigenvector coresponding to $\lambda = 40$
    
\begin{flalign*}
    \boldsymbol{\lambda = 40}: 
    \end{flalign*}
    
    $$
    (\mathbf{A} - 10\mathbf{I})\vec{Y} = \left[\begin{array}{ll}
    (25-10) & -15  \\
    -15 & (25-10) 
    \end{array}\right]\begin{bmatrix}
    y_1 \\
    y_2
    \end{bmatrix} = \mathbf{0}
    $$
    
    Solving the system of equations:
        \begin{equation}
            \begin{cases}
              \phantom{-}15y_1 - 15y_2 &= 0 \\
              -15y_1 + \phantom{-}15y_2 &= 0
            \end{cases}
        \end{equation}
    
    which results in $y_2 = -y_1$. Thus, for any real number $y_1$, a vector of the form
        $\vec{X} = \left[\begin{array}{r}
                y_1  \\
                y_1
               \end{array}\right]$
    
    is an eigenvector of $\mathbf{A^TA}$ corresponding to $\lambda = 10$.
    To make it an unit length divide by its magnitude  \\
    $$\left\lVert Y \right\rVert = 
    \sqrt{y_1^2 + y_2^2} = \sqrt{2y_1^2} = \sqrt{2}y_1
    $$
    
    Hence, $\hat{Y} = \left[\begin{array}{l}
                           \frac{1}{\sqrt{2}}  \\
                            \frac{1}{\sqrt{2}} 
                           \end{array}\right]$
    is the unit eigenvector coresponding to $\lambda = 10$

    Thus the matrix $\mathbf{V} = \begin{bmatrix} \hat{X}, \hat{Y}\end{bmatrix}$
is  given by:
\[
\mathbf{V} = \left[\begin{array}{ll}
\phantom{-}\frac{1}{\sqrt{2}} & \frac{1}{\sqrt{2}}  \\
-\frac{1}{\sqrt{2}} & \frac{1}{\sqrt{2}}
\end{array}\right]
\]
and its transpose is: 
$$
\mathbf{V^T} = \left[\begin{array}{ll}
\frac{1}{\sqrt{2}} & -\frac{1}{\sqrt{2}}  \\
\frac{1}{\sqrt{2}} & \phantom{-}\frac{1}{\sqrt{2}}
\end{array}\right]
$$

\section{Construct $\mathbf{U} = \mathbf{A}\mathbf{V}\bf{\Sigma}^{-1}$}

    \begin{flalign*}
    &\mathbf{U} = \left[\begin{array}{ll}
    4  & \phantom{-}0  \\
    3  & -5
    \end{array}\right] \left[\begin{array}{ll}
    \phantom{-}\frac{1}{\sqrt{2}} & \frac{1}{\sqrt{2}}  \\
    -\frac{1}{\sqrt{2}} & \frac{1}{\sqrt{2}}
    \end{array}\right] \left[\begin{array}{ll}
    0.1581 & 0  \\
    0 & 0.3162
    \end{array}\right] = \left[\begin{array}{ll}
    4  & \phantom{-}0  \\
    3  & -5
    \end{array}\right] \left[\begin{array}{ll}
    \phantom{-}\frac{\sqrt{5}}{20} & \frac{\sqrt{5}}{10}  \\
    -\frac{\sqrt{5}}{20} & \frac{\sqrt{5}}{10}
    \end{array}\right] = \left[\begin{array}{ll}
    \frac{\sqrt{5}}{5} & \phantom{-}\frac{2\sqrt{5}}{5}  \\
    \frac{2\sqrt{5}}{5} & -\frac{\sqrt{5}}{5}
    \end{array}\right]  \approx \left[\begin{array}{ll}
    0.4472  & \phantom{-}0.8944  \\
    0.8944  & -0.4472
    \end{array}\right] &
    \end{flalign*}

\section{Check if we get the matrix A from the matrix multiplication}

\begin{flalign*}
    \mathbf{U} &= \left[\begin{array}{ll}
    0.4472  & \phantom{-}0.8944  \\
    0.8944  & -0.4472
    \end{array}\right] \left[\begin{array}{ll}
    6.3246 & 0  \\
    0 & 3.1623
    \end{array}\right] \left[\begin{array}{ll}
    0.7071 & -0.7071  \\
    0.7071 & \phantom{-}0.7071
    \end{array}\right] = \left[\begin{array}{ll}
    0.4472  & \phantom{-}0.8944  \\
    0.8944  & -0.4472
    \end{array}\right] \left[\begin{array}{ll}
    4.4721 & -4.4721  \\
    2.2361 & 2.2361
    \end{array}\right] \\ 
    &= \left[\begin{array}{ll}
    3.9998 & 0.00004  \\
    2.9999 & -4.9998
    \end{array}\right]  \approx \left[\begin{array}{ll}
    4  & \phantom{-}0  \\
    3  & -5
    \end{array}\right] = \mathbf{A} &
    \end{flalign*}


\end{document}