\documentclass[14pt]{article}

\usepackage{amsmath}
\usepackage[colorlinks=true, allcolors=blue]{hyperref}

\title{Milestone 1: Eigenvalues and Eigenvectors of a Matrix}
\author{Ellie}

\begin{document}
\maketitle

\section{Compute the eigenvalues and eigenvectors of the matrix}
 
  $$
    A = \left[\begin{array}{ll}
    8 & 3  \\
    2 & 7 
    \end{array}\right]
  $$

\section{Find the determinant of $\det(\mathbf{A} - \lambda\mathbf{I})$}
   $$
    \det(\mathbf{A} - \lambda\mathbf{I}) = \left|\begin{array}{ll}
    8-\lambda & 3  \\
    2 & 7-\lambda 
    \end{array}\right|
    $$
    $$
    = (8-\lambda)(7-\lambda) - 6
    $$
    $$
    = \lambda^2 - 15\lambda + 50
    $$

\section{Solve for the eigenvalues}
    $$
    \lambda^2 - 15\lambda + 50 = 0
    $$
    $$
    \lambda = \frac{-(-15) \pm \sqrt{(-15)^2 - 4(1)(50)}}{2(1)}
    $$
    $$
    = \frac{15 \pm \sqrt{225 - 200}}{2}
    $$
    $$
    = \frac{15 \pm \sqrt{25}}{2}
    $$
    $$
    = \frac{15 \pm 5}{2}
    $$
    $$
    = 10, 5
    $$

\section{Solve for the eigenvectors}

\begin{flushleft}
For $\boldsymbol{\lambda = 10}$:
\end{flushleft}

$$
(\mathbf{A} - 10\mathbf{I})\vec{x} = \left[\begin{array}{ll}
-2 & \phantom{-}3  \\
\phantom{-}2 & -3 
\end{array}\right] \begin{bmatrix}
x_1 \\
x_2
\end{bmatrix} = \mathbf{0}
$$


Solving the system of equations:
    \begin{equation}
        \begin{cases}
          -2x_1 + 3x_2 &= 0 \\
          \phantom{-}2x_1 - 3x_2 &= 0
        \end{cases}
    \end{equation}

gives $x_1$ being any real number and $x_2 = \frac{2}{3}x_1$. Thus any vector of the form
    
    $\left[\begin{array}{r}
             x_1  \\
              \frac{2}{3}x_1
           \end{array}\right]$ is an eigenvector for $\lambda = 10$.  
A specific example is $\mathbf{x} = \left[\begin{array}{l}
                        3  \\
                        2 
                       \end{array}\right]$
    
\vspace{2\baselineskip}
\begin{flushleft}
For $\boldsymbol{\lambda = 5}$:
\end{flushleft}
    $$
    (\mathbf{A} - 5\mathbf{I})\vec{y} = \left[\begin{array}{ll}
        3 & 3  \\
        2 & 2 
    \end{array}\right] \begin{bmatrix}
        y_1 \\
        y_2
    \end{bmatrix} = \mathbf{0}
    $$

Solving the system of equations:

    $$
        \begin{cases}
          3y_1 + 3y_2 = 0 \\
          2y_1 + 2y_2 = 0
        \end{cases}
    $$

    gives $y_1$ being any real number and $y_2$ = $- y_1$. Thus, any vector of the form 

    $\left[\begin{array}{r}
    y_1  \\
    - y_1
    \end{array}\right]$ where $y_1$ is a any real number, is an eigenvector for $\lambda = 5$.\\ 
    
    A specific example is  $$ \mathbf{y} = \left[\begin{array}{r}
                                            1  \\
                                            -1 
                                \end{array}\right]
                            $$


\end{document}